\documentclass[12pt]{article}
\usepackage{graphicx}
\usepackage{url}
\usepackage{hyperref}

\usepackage{xcolor}

\usepackage{color}
\usepackage{listings}
\usepackage{courier}

\lstset{
         basicstyle=\footnotesize\ttfamily, % Standardschrift
         %numbers=left,               % Ort der Zeilennummern
         numberstyle=\tiny,          % Stil der Zeilennummern
         %stepnumber=2,               % Abstand zwischen den Zeilennummern
         numbersep=5pt,              % Abstand der Nummern zum Text
         tabsize=2,                  % Groesse von Tabs
         extendedchars=true,         %
         breaklines=true,            % Zeilen werden Umgebrochen
         keywordstyle=\color{red},
    		frame=b,         
 %        keywordstyle=[1]\textbf,    % Stil der Keywords
 %        keywordstyle=[2]\textbf,    %
 %        keywordstyle=[3]\textbf,    %
 %        keywordstyle=[4]\textbf,   \sqrt{\sqrt{}} %
         stringstyle=\color{white}\ttfamily, % Farbe der String
         showspaces=false,           % Leerzeichen anzeigen ?
         showtabs=false,             % Tabs anzeigen ?
         xleftmargin=17pt,
         framexleftmargin=17pt,
         framexrightmargin=5pt,
         framexbottommargin=4pt,
         %backgroundcolor=\color{lightgray},
         showstringspaces=false      % Leerzeichen in Strings anzeigen ?        
 }
 \lstloadlanguages{% Check Dokumentation for further languages ...
         %[Visual]Basic
         %Pascal
         %C
         %C++
         %XML
         %HTML
         Java
 }
    %\DeclareCaptionFont{blue}{\color{blue}} 

  %\captionsetup[lstlisting]{singlelinecheck=false, labelfont={blue}, textfont={blue}}
\usepackage{caption}
\DeclareCaptionFont{white}{\color{white}}
\DeclareCaptionFormat{listing}{\colorbox[cmyk]{0.43, 0.35, 0.35,0.01}{\parbox{\textwidth}{\hspace{15pt}#1#2#3}}}
\captionsetup[lstlisting]{format=listing,labelfont=white,textfont=white, singlelinecheck=false, margin=0pt, font={bf,footnotesize}}


\begin{document}

\title{Interfacing between R and Python}
\author{
        Hoofar Pourzand \\
        Research Computing and Cyberinfrustructure Group\\
        Pennsylvania State University\\        
            \and
        %William Brower\\
        %RCC Group, Advanced Computing Center\\
        }
\date{\today}
\maketitle

\begin{abstract}
R has a rich developing community and consequently a growing list of libraries, methods for data mining \& statistical analysis of data. Also, Python is a popular interpreted language used in many back-end and front end web-development technologies, as well as data mining tools \& libraries that easily incorporates Object oriented design concepts into a code. Calling R from Python, for a very simple example is shown here to give Python the extra computational edge that R already provides.\\
RPy is the interface between R and Python which is used for this task. 
\end{abstract}

\section{Preparation}

\paragraph{Installing RPy}
If you are running Ubuntu, simply type: \\
\begin{lstlisting}[label=Adding RPy ,caption= Installing RPy]
$sudo apt-get install python-rpy
\end{lstlisting}

To load RPy from Python, whether in interactive mode or in Batch (PBS) mode, to your python script just add:
\begin{lstlisting}[label=Loading RPy ,caption= Loading PPy]
>>>from rpy import*
\end{lstlisting}
This will load a Python class instance r.

\paragraph{Running RPy}
From Python prompt type in:
\begin{lstlisting}[label=Adding RPy ,caption= Running RPy]
>>>r.hsit(r.rnorm(10), main= '', xlab= '')
>>>a = [5,12, 13]
>>>b= [10,28,30]
>>>lmout = r.lm('v2~v1' , data = r.data_frame(v1=a, v2= b))

\end{lstlisting}

Note R function names are prefixed by \textit{r.} .
Note Python does not include a tilde character, in these cases we need to specify the model formula via a string. For calling R functions in Python that have a period in their R function name, an underscore is substituted, e.g. $data\_frame() 
$
The output object is a Python Dictionary- closest to an R list type.\\
To access the results accordingly, just type in the attributes title:\\
\begin{verbatim}
lmout['coefficient']
>>>lmout[coefficients']['v1']
\end{verbatim}
To avoid syntax clashes between Python and R, you can submit R commands to work on R name-spaces by using the function r().

\begin{lstlisting}[label=example ,caption= Calling Python in R Example]
>>>r.library('lattice')
>>>r.assign('a',a)           ***copy a variable from Python's namespace to R.
>>>r.assign('b', b)
>>>r('g <- expand.grid(a,b)')   *** assigning the result to g in R's namespace.
>>>r('g$Var3 <- g$Var1^2 +g$Var1 * g$Var2')
>>>r('wireframe(Var3~Var1 + Var2, g)')
>>>r('plot(wireframe(Var3 ~Var1 +Var2, g)))      *** wireframe() didn't diplay.

\end{lstlisting}

\bibliographystyle{}
\bibliography{}

\begin{thebibliography}{}
\bibitem{}
\url{http://rpy.sourceforge.net/rpy/doc/rpy.pdf}
\bibitem{}
\url{http://www.daimi.au.dk/~besen/TBiB2007/lecture-notes/rpy.html}

\end{thebibliography}

 
\end{document}
